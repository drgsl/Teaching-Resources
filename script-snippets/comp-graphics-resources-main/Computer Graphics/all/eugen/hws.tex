\documentclass[landscape]{article}

\usepackage[margin=0.1cm]{geometry}

\usepackage[cmex10]{amsmath}
\usepackage{amsfonts}


%\usepackage{mdwmath}
%\usepackage{mdwtab}
%\usepackage{fixltx2e}
%\usepackage{stfloats}
%\usepackage{multirow}
%\usepackage{attachfile}
%\usepackage{xargs}
%\usepackage{filecontents}
%%\usepackage{datatool-pgfmath}
%\usepackage{datatool}
%\DTLsetseparator{ }
%\usepackage{xstring}
%\usepackage{ifthen}

\usepackage{graphicx, tikz, pgfplots}
\pgfplotsset{compat=1.14}
\usetikzlibrary{calc}
\usetikzlibrary{plotmarks}
%\usetikzlibrary{pgfplots.groupplots}
\usepgfplotslibrary{polar,colormaps,groupplots}

\usepackage{listings} %for code display

\usepackage{multicol}

%\selectcolormodel{cmyk}
%\interdisplaylinepenalty=2500
\begin{document}
\begin{multicols}{2}

  \def\beglst{\begin{lstlisting}[breaklines=true, language=TeX, morekeywords={\begin, \draw, --, rectangle, circle, cycle, \def, \foreach, \addplot}]}
  
\beglst
\begin{tikzpicture}[scale = 6]
  \draw[black] (-1, -1) rectangle (1, 1);
  \draw[blue] (-1, 1) -- (1, -1);
  \draw[red] (-1, -1) -- (1, 1);
  \draw[red] (-0.5, 0) -- (0.5, 0);
\end{tikzpicture}
\end{lstlisting}


tikzpicture is the environment we need to place TikZ images in. The scale=6 parameter enlarges (many of the elements in) the image by a factor of 6 (since it was originally too small).
draw is the basic TikZ command. Starting at the (-1, -1) coordinate, we draw a rectangle to (1,1);
Next, we draw different lines using the -- element.
All draw and similar commands need to be ;-terminated.

\begin{tikzpicture}[scale = 6]
  \draw[black] (-1, -1) rectangle (1, 1);
  \draw[blue] (-1, 1) -- (1, -1);
  \draw[red] (-1, -1) -- (1, 1);
  \draw[red] (-0.5, 0) -- (0.5, 0);
\end{tikzpicture}

\columnbreak

\beglst
\begin{tikzpicture}[scale = 6]
  \draw[black] (-1, -1) rectangle (1, 1);
  \draw[red, thick = 5] (1, 1) -- (1, 0.5) -- (0.5, 0.1) -- (0.5, -0.3) -- cycle;
  \draw[orange, thick = 3] (1, 1) circle(0.05);
\end{tikzpicture}
\end{lstlisting}


We can draw multiple lines in a single draw commands. We can loop back using \textbf{cycle}. Line joints are drawn at sharp angles, and they may exceed the basic square we've drawn (as shown by the orange circle).

\begin{tikzpicture}[scale = 6]
  \draw[black] (-1, -1) rectangle (1, 1);
  \draw[red, thick = 5] (1, 1) -- (1, 0.5) -- (0.5, 0.1) -- (0.5, -0.3) -- cycle;
  \draw[orange, thick = 3] (1, 1) circle(0.05);
\end{tikzpicture}

\columnbreak

\beglst
\def \hexagon#1#2 {
  \fill[#2] (0: #1) -- (60: #1) -- (120: #1) -- (180: #1) -- (240: #1) -- (300: #1) -- cycle;
}
\begin{tikzpicture}[scale=6]
  \draw[black] (-1, -1) rectangle (1, 1);
  \hexagon{0.8}{blue}
  \hexagon{0.51}{red}
  \hexagon{0.5}{white}
\end{tikzpicture}
\end{lstlisting}

\textbf{{\textbackslash}def} is the most basic way of defining a macro. The hexagon macro has two arguments. Using the \textbf{{\textbackslash}fill} command, it fills the interior of a geometric figure. The vertices of the hexagon are specified using polar coordinates. Due to issues combining \textbf{{\textbackslash}foreach} and \textbf{{\textbackslash}draw}, the vertex angular component is specified by hand. An alternative is using TikZ lists to programatically draw the hexagon.


% macro definition
\def \hexagon#1#2 {%
  \fill[#2] (0: #1) -- (60: #1) -- (120: #1) -- (180: #1) -- (240: #1) -- (300: #1) -- cycle;
  %foreach inside draw commands is very very very buggy. So, I had to generate these numbers by hand. A list could do, too.
}

\begin{tikzpicture}[scale=6]
  \draw[black] (-1, -1) rectangle (1, 1);
  \hexagon{0.8}{blue}
  \hexagon{0.51}{red}
  \hexagon{0.5}{white}
\end{tikzpicture}


\pagebreak

\beglst
\begin{tikzpicture}[scale = 6]
  \draw[black] (-1, -1) rectangle (1, 1);
  \def\xmax{20}
  \def\border{1.1}
  \draw[red] (0, {1 / \border}) -- ({0.1 / (\xmax * \border)}, {1 / \border});
  \draw[red] \foreach \x [remember=\x as \px (initially 0.1)] in {0.1,0.2,...,\xmax}{
    ({\x / (\xmax * \border)}, {abs(round(\x) - \x)) / (\x * \border)}) -- ({\px / (\xmax * \border)}, {abs(round(\px) - \px) / (\px * \border)})
  };%don't forget the ; when we exit the loop and return to the \draw environment
\end{tikzpicture}
\end{lstlisting}

Using \textbf{{\textbackslash}foreach}, we generate a set of points. Then, we draw lines between each successive two points. \{\} are used to evaluate more complex mathematical expressions.

\begin{tikzpicture}[scale = 6]
  \draw[black] (-1, -1) rectangle (1, 1);
  \def\xmax{20}
  \def\border{1.1}
  \draw[red] (0, {1 / \border}) -- ({0.1 / (\xmax * \border)}, {1 / \border});
  \draw[red] \foreach \x [remember=\x as \px (initially 0.1)] in {0.1,0.2,...,\xmax}{
    ({\x / (\xmax * \border)}, {abs(round(\x) - \x)) / (\x * \border)}) -- ({\px / (\xmax * \border)}, {abs(round(\px) - \px) / (\px * \border)})
  };%don't forget the ; when we exit the loop and return to the \draw environment
\end{tikzpicture}

\columnbreak

\beglst
\def\plotfunc#1#2#3#4{%x, xmax, ymax, border
  ({#1 / (#2 * #4)}, {\func{#1} / (#3 * #4)})%
}

\begin{tikzpicture}[scale = 5.5]
  \draw[black] (-1, -1) rectangle (1, 1);
  \def\xmax{20}
  \def\ymax{1}
  \def\border{1.1}
  \def\func#1{%
    abs(round(#1) - #1) / #1%
  }
  \draw[red] (0, {1 / (\ymax * \border)}) -- ({0.1 / (\xmax * \border)}, {1 / (\ymax * \border)});
  \draw[red] \foreach \x [remember=\x as \px (initially 0.1)] in {0.1,0.2,...,\xmax}{
    \plotfunc{\px}{\xmax}{\ymax}{\border} -- \plotfunc{\x}{\xmax}{\ymax}{\border}
  };
\end{tikzpicture}
\end{lstlisting}

The same image, but now we have a macro-wrapper for drawing $y = f(x)$-type functions. The function itself is defined inside the \textbf{tikzpicture} environment.

%working with macro functions now!
\def\plotfunc#1#2#3#4{%x, xmax, ymax, border
  ({#1 / (#2 * #4)}, {\func{#1} / (#3 * #4)})%
}

\begin{tikzpicture}[scale = 5.5]
  \draw[black] (-1, -1) rectangle (1, 1);
  \def\xmax{20}
  \def\ymax{1}
  \def\border{1.1}
  \def\func#1{%
    abs(round(#1) - #1) / #1%
  }
  \draw[red] (0, {1 / (\ymax * \border)}) -- ({0.1 / (\xmax * \border)}, {1 / (\ymax * \border)});
  \draw[red] \foreach \x [remember=\x as \px (initially 0.1)] in {0.1,0.2,...,\xmax}{
    \plotfunc{\px}{\xmax}{\ymax}{\border} -- \plotfunc{\x}{\xmax}{\ymax}{\border}
  };
\end{tikzpicture}

\columnbreak

\beglst
\begin{tikzpicture}[scale=1.5]
  \begin{axis}[
      axis equal,
      axis lines=middle,
      axis line style={->},
      tick style={color=black},
      xtick=\empty,
      ytick=\empty,
      xmin=-1, xmax=1,
      ymin=-1, ymax=1
    ]
    \def\a{0.3}
    \def\b{0.2}
    \addplot[
      color=red,
      samples=100,
      domain=-pi:{pi+0.1},
      variable=\t
    ]
    ({2 * (\a * cos(deg(t)) + \b) * cos(deg(t))}, {2 * (\a * cos(deg(t)) + \b) * sin(deg(t))});
  \end{axis}
\end{tikzpicture}

\end{lstlisting}

This image uses the \textbf{pgfplots} package. It provides the \textbf{axis} environment inside the \textbf{tikzpicture} environment.

When adding a specific plot inside an axis environment, we can specify the samples, the domain (from which the ratio arises), and we can also change the variable. Below, we can draw a function where $x = f_x(t)$ and $y = f_y(t)$. By default, the pgfplot scale is mismatched with the default tikzpicture scale; this is why we need to upscale this image less.

\begin{tikzpicture}[scale=1.5]
  \begin{axis}[
      axis equal,
      axis lines=middle,
      axis line style={->},
      tick style={color=black},
      xtick=\empty,
      ytick=\empty,
      xmin=-1, xmax=1,
      ymin=-1, ymax=1
    ]
    \def\a{0.3}
    \def\b{0.2}
    \addplot[
      color=red,
      samples=100,
      domain=-pi:{pi+0.1},
      variable=\t
    ]
    ({2 * (\a * cos(deg(t)) + \b) * cos(deg(t))}, {2 * (\a * cos(deg(t)) + \b) * sin(deg(t))});
  \end{axis}
\end{tikzpicture}

%%pgfplots are rather buggy with more complex plots. Fortunately, it allows us to use gnuplot to compute the points, and pgfplots will serve as a wrapper to nicely integrate the plot in our document. Uncomment the following if you have gnuplot installed. Run pdflatex with the -shell-escape argument, allowing it to run the external program gnuplot. Also, the following image might take a while to generate.
%   \begin{tikzpicture}
%     \begin{axis}[width=\linewidth, xlabel=$x_1$, ylabel=$x_2$, zlabel=f(X), title={Michalewicz's Function}, samples=101, enlargelimits=false, 3d box, xmin=0, ymin=0$
%       \addplot3[
%         contour gnuplot={
%           % cdata should not be affected by z filter:
%           output point meta=rawz,
%           number=10,
%           labels=false,
%         },
%         samples=41,
%         z filter/.code=\def\pgfmathresult{-2},
%       ] 
%       gnuplot [raw gnuplot] {%
%         set xrange [0:pi];
%         set yrange [0:pi];
%         set isosample 101,101;
%         splot -sin(x) * sin(x**2/pi) ** 20 - sin(y) * sin(2 * y**2 / pi)**20; };
%
%       \addplot3[surf] gnuplot [raw gnuplot] {%
%         set xrange [0:pi];
%         set yrange [0:pi];
%         set isosample 101,101;
%         splot -sin(x) * sin(x**2/pi) ** 20 - sin(y) * sin(2 * y**2 / pi)**20; };
%
%     \end{axis}
%   \end{tikzpicture}



\end{multicols}
\end{document}


